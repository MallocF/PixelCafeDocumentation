\documentclass[a4paper, 10pt]{article}

\usepackage[margin = 1in]{geometry} % for spacing around
\usepackage{graphicx} % for including images in your pdfs
\usepackage{xcolor} % for including colors in your pdf
\usepackage{soul} % for text decoration
\usepackage[utf8]{inputenc} % for encoded text
\usepackage[T1]{fontenc}
\usepackage{setspace} % for setting different line spacings between paragrafs.
\usepackage{enumerate} % for letting us get more detailed enumerate lists
\usepackage{multirow} % to let us combine more rows together
\usepackage{colortbl} % for decorating tables
\usepackage{amsmath} % used for representing more complicated math displays
\usepackage{supertabular}
\usepackage{longtable} % both of these packages are used to making really big tables
\usepackage{wrapfig} % allows us to wrap text around figures
\usepackage{fancyhdr} % for making fancy headers
%\usepackage{bibtex} % for making better bibliographies
\usepackage[pdftex]{hyperref} % for letting us make links
\usepackage{lscape} % Allows us to flip from portrait to landspace
\usepackage{tikz} % for high detailed drawing
\usepackage{multicol} % To put things side by side
\usepackage{rotating} % For rotating objects
\usepackage{hyperref} % for hyper links

% Setting up the default image path
\graphicspath{{./Pictures/}}


% Setting up the fancy page style
\fancypagestyle{customStyle}{
    \lhead{} \chead{} \rhead{}
	\lfoot{} \cfoot{\thepage} \rfoot{}
	\renewcommand{\headrulewidth}{0pt}
	\renewcommand{\footrulewidth}{1pt}
}
\pagestyle{customStyle}
    
% Predefining custom colors
\definecolor{importantColor}{RGB}{255, 0, 0}
\definecolor{mediocerColor}{RGB}{255, 150, 37}
\definecolor{moneyColor}{RGB}{123, 230, 29}
\definecolor{noteColor}{RGB}{14, 192, 208}
    
% Setting up hyperref options
\hypersetup {
    colorlinks = false,
    citecolor = black,
    filecolor = blue,
    linkcolor = blue,
    urlcolor = blue,
    pdftex
}

% Custom commands
\newcommand{\itemizeListDisplay}[1]{\sc{#1}}
\newcommand{\colorLink}[3]{\hyperlink{#2}{\textcolor{#1}{#3}}}
\newcommand{\linkSection}[1]{\hypertarget{Section:#1}{\section{#1}}}
\newcommand{\subtitle}[1]{\begin{large}{\textbf{#1}}\end{large}}
\newcommand{\fakeCaption}[1]{
    \vspace{2mm}
    \noindent #1
}
\newcommand{\displayRecepie}[2]{
    \noindent #1: \newline{}
    \noindent \hspace{10mm} #2
}

% Implementing the author detail
\title{Pixel Cafe Game Documentation}
\author{Aidan Abdulov}
\date{}

% Start work on the document
\begin{document}
	\maketitle
    \vspace{5mm}

	\begin{abstract}
	    \noindent This document was made for the soul reason to plan out the upcoming game "Pixel Cafe" on Roblox.
	    This document will mention everything from how the main menu should be designed, the npc interactions and logic
	    map plans and design, and anything else needed to help aid the development team to success.
    \end{abstract}
	\pagebreak

	\tableofcontents
	\pagebreak

    \linkSection{Main Menu}

    \subsection{Scenery shot}
    \noindent Upon a player joining after they are done loading their camera should be placed looking at an
    location of an \emph{custom build} map. Adding blurs and other visual effects is \textcolor{mediocerColor}{optional}.
    Their should be multiple versions of the map depending on the season, ie if it's winter a snowy version of the map
    should be loaded in, and if it's summer or some other season or an event like hallowen the map should be loaded to fit the mood.

    \vspace{2mm}
    \noindent For more info about the map design please click \colorLink{noteColor}{Section:Map Design}{here}
    \subsection{UI for the Main Menu}
    \noindent The next list will show all the options the main menu will have to have...

    \begin{itemize}
        \item \itemizeListDisplay{Party UI}
        \item \itemizeListDisplay{Store UI}
        \item \itemizeListDisplay{Trade UI}
        \item \itemizeListDisplay{Inventory UI}
        \item \itemizeListDisplay{Leaderboard / Rank UI}
        \label{list:MainMenuUI}
    \end{itemize}

    \subsubsection{Party UI}

    \noindent The party UI is a group of UI's that should give the player options to:
    \vspace{5mm}
    \begin{itemize}
        \item \itemizeListDisplay{Create parties}
        \item \itemizeListDisplay{Invite players to parties}
        \item \itemizeListDisplay{Remove players from parties}
        \item \itemizeListDisplay{Start a game with your party}
        \item \itemizeListDisplay{Join friend}
    \end{itemize}

    \vspace{5mm}
    \noindent When creating parties the party maker/creator should be allowed to make it an \textcolor{green}{opened} or
    \textcolor{red}{closed} party. If the party is open then it will be displayed for everyone to see and join, however if the party is closed then
    only the party creator can invite other players that \textcolor{importantColor}{aren't} in a party.
    \textcolor{importantColor}{Only the party creator} can kick / remove players from his party. And finally
    upon pressing play the game should teleport everyone in the party to the desired location, if the teleport fails
    the players are to be informed of the failed teleport. When creating a party the player will be presented with more then just one
    option to make the party opened or closed. The next list will show you all the options the player will be promted upon making a party.
    \vspace{2mm}

    \begin{itemize}
        \item \itemizeListDisplay{Party Type}
        \item \itemizeListDisplay{Friends can join session}
    \end{itemize}

    \vspace{3mm}
    \noindent If the party is an open type it will get a few more \textcolor{moneyColor}{bonus} option... such as
    \begin{itemize}
        \item \itemizeListDisplay{Minimum Level Requierment}
        \item \itemizeListDisplay{Minimum Days Survived}
    \end{itemize}

    \noindent Keep in mind that all of these bonus options are \textcolor{importantColor}{OPTIONAL}.
    \pagebreak

    \subsubsection{Store and Trade UI's}
    \noindent Before reading about these UI's it will come in \emph{handy} if you check out the \colorLink{noteColor}{Section:Player Data}{Player Data} section.
    \vspace{2mm}

    \noindent The store UI should allow the player to purchase the listed items:
    \vspace{3mm}

    \begin{itemize}
        \item \itemizeListDisplay{In game ``loot crates''}
        \item \itemizeListDisplay{Gamepass's}
        \item \itemizeListDisplay{Dev Products}
    \end{itemize}

    \vspace{2mm}
    \noindent To learn more about all of these please click \colorLink{noteColor}{Section:Monetization}{here}.\newline{}
    \noindent The Shop UI should offer the player options to select what they wanna buy, on the far right side it should
    display the item the player selected AND \emph{Show the items description}, it should also show you the items price,
    and the currency you would need to pay for the item.

    \vspace{2mm}
    \begin{spacing}{1.2}
        \noindent Once the player press's the \emph{trade button}, they will be promted with a player list that they can requests trades with.
        Upon sending a request if the player accept the request a new UI would be represented that will let the players put stuff from their inventory to the trading UI.
        The trading UI will be a giant frame, on the top of the frame will be a $6\times2$ grid, the first 6 slots will be for the first player and the other 6 for the other player.
        If they have multiple of the item they would be promoted with an option to select how much of \emph{that one item} should they put into the trade, this means they can have up to 6 \textcolor{importantColor}{\emph{unique}} items in the trade.
        On the right side of the main UI will be the other persons character frame with their name below them, and right below that would be a private chat between the two players.
        Below the trading part would be the players inventory with a filter bar on the top of that inventory frame, and right below the inventory will be an \textcolor{moneyColor}{accept} and \textcolor{red}{decline} button.    
    \end{spacing}

    \linkSection{Main Game}
    \noindent This section will talk about the main game mechanics.

    \subsection{Basic Gameplay Overview}
    \begin{spacing}{1.2}
        \noindent Once all the players get teleported and the game starts the players will need to wait for the
        party leader to chose a save file, once a save file has been selected and the entire cafe loaded the players
        will also get teleported into the cafe. If the game is brand new they get teleported into the cafe with all the default tools.
        The game will start off with a small storage unit, bakery, and the main cafe area, it will also start off with a small parking space
        which will be important to upgrade since the more parking space you got the more room for people to come you got. The game runs over two basic systems,
        money and ratings. If at any time your rating goes way below the \emph{requiered minimum rating} for staying open or if you do not manage to make
        enough money to pay daily taxes you will lose the game and the save file will be forcfully deleted. The game has a few different
        things that need to be managed by players, one of the things will be serving customers, a player is expected to go to a custom, talk to them and
        write everything down, upon writing everything down you will give the order to the barista who would take care of the order, and there are planty of other jobs
        that you can take a look at \hyperlink{Sub Section:Jobs}{\textcolor{noteColor}{here}}. Another aspect of the game has to do with the fact
        that food can and will get spoiled, and also ordering ingridients takes a lot of time, so making sure you order enough and order in things in time will be really important,
        upon the day finishing if the players haven't lost they will be promted with a continiue button so the players can have a breather or if they need to take a bathroom break.
        And that's the \emph{brief} overview of the game, for more detail on everything else please keep reading the document.
    \end{spacing}

    \subsection{Player Interactions}
    \begin{spacing}{1.2}
        \noindent Players will be allowed to take tools / items with them. Upon taking an item it will take a slot
        either in your right or left hand depending on which hand u used to pick it up. There are \emph{two} types of things you can pick up,
        light things [such as knives, forks, etc], and heavy things [such as chairs, tables, etc]. When carrying an heavy thing
        \textcolor{importantColor}{\emph{both}} of your hands are requiered to be free. Certain things such as putting food on an
        tray can help you carry multiple things at once.
    \end{spacing}
    \vspace{2mm}

    \hypertarget{Sub Section:Jobs}{\subsection{Jobs}}
    \subsubsection{Drinks Assembling}
        \noindent Drink assembling as the name says, is a job that requiers someone to assembly drinks, but before
        we can explain the process that has to be done, let's first off start listing everything we need for the drink assembler to do his/her job:
        \vspace{2mm}

        \begin{enumerate}
            \item Blender
            \item Grinding Machine
            \item Steamer
            \item Ice Maker
            \item Cup \& Lid Holder
            \item Syrup Holders / Dispensers
            \item Cream Holders / Dispensers
            \item Cutting Board
            \item Juicer
        \end{enumerate}

        \begin{center}
            \fakeCaption{\sc{List of tools}}
        \end{center}

        \pagebreak
        \begin{itemize}
            \item 
                \noindent \itemizeListDisplay{Syrups}
                \begin{enumerate}
                    \item Caramel
                    \item Chocolate
                    \item Vanilla
                    \item Honey
                \end{enumerate}
            \item 
                \noindent \itemizeListDisplay{Creams}
                \begin{enumerate}
                    \item Caramel
                    \item Chocolate
                    \item French Vanilla
                    \item Hazlenut
                    \item Coconut
                    \item Almond
                    \item Peppermint
                    \item Pumpkin Spice
                \end{enumerate}
            \item 
                \noindent \itemizeListDisplay{Milk}
                \begin{enumerate}
                    \item Skin Milk
                    \item Whole Milk
                    \item Soy Milk
                    \item Almond Milk
                    \item Coconut Milk
                \end{enumerate}
            \item
                \noindent \itemizeListDisplay{Toppings}
                \begin{enumerate}
                    \item Cookies Crumbs
                    \item Chocolate Chips
                    \item Sprinkles
                    \item Cherrys
                    \item Lemons
                    \item Strawberries
                    \item 
                        \noindent Powders
                        \begin{enumerate}
                            \item Chocolate Powder
                            \item Vanilla Powder
                        \end{enumerate}
                \end{enumerate}
            \item 
                \noindent \itemizeListDisplay{Coffee Beans}
                \begin{enumerate}
                    \item Brazilian Beans
                    \item French Vanilla Beans
                    \item Dark Roast Beans
                    \item Hazelnut Beans
                \end{enumerate}
            \item
                \noindent \itemizeListDisplay{Tea Leaves}
                \begin{enumerate}
                    \item Black Tea
                    \item Green Tea
                    \item White Tea
                    \item Yellow Tea
                    \item Apple Tea
                    \item Turkish Tea
                    \item Raspberry Tea
                \end{enumerate}
        \end{itemize}

        \begin{center}
            \fakeCaption{\sc{List of ingridients}}
        \end{center}
        \pagebreak

        \noindent \subtitle{Frappes}

        \begin{enumerate}
            \item \displayRecepie{ChocoMoco Frappe}{Hazlenut beans, w/ chocolate creamer, blended with ice. Then topped with whipping cream and chocolate syrup, finally, you will sprinkle some chocolate chips ontop of the whipping cream.}
            \item \displayRecepie{Caramel Blessing Frappe}{Steamed water, Brazilian beans, w/ caramel creamer, blended with ice. Then topped with whipping cream and Caramel syrup.}
            \item \displayRecepie{Vanilla Lovers Frappe}{Steamed water,  French vanilla roast beans, w/ French vanilla creamer, blended with Ice. Then topped with whipping cream and vanilla syrup, finally, you will sprinkle some vanilla powder ontop of the whipping cream.}
        \end{enumerate}

        \noindent \subtitle{Espresso}

        \begin{enumerate}
            \item \displayRecepie{Brazilian's Espresso}{Brazilian beans w/ steamed water and sugar}
            \item \displayRecepie{French Vanilla Espresso}{French vanilla beans w/ steamed water and sugar}
            \item \displayRecepie{Dark Roast Espresso}{Dark Roast beans w/ steamed water and sugar}
            \item \displayRecepie{Hazelnut Espresso}{Hazelnut beans w/ steamed water and sugar}
        \end{enumerate}

        \noindent \subtitle{Americano}
        
        \begin{enumerate}
            \item \displayRecepie{Brazilian's Espresso}{Brazilian beans w/ steamed water, more added water w/ sugar}
            \item \displayRecepie{French Vanilla Espresso}{French vanilla beans w/ steamed water, more added water w/sugar}
            \item \displayRecepie{Dark Roast Espresso}{Dark Roast beans w/ steamed water, more added water w/sugar}
            \item \displayRecepie{Hazelnut Espresso}{Hazelnut beans w/ steamed water, more added water w/sugar}
        \end{enumerate}

        \noindent \subtitle{Cappuchino}

        \begin{enumerate}
            \item \displayRecepie{Amazonian rain}{Brazilian beans, w/ steamed coconut milk. Topped with Milk foam}
            \item \displayRecepie{Vanilla perfection}{French vanilla beans, w/ steamed whole milk. Topped with milk foam}
            \item \displayRecepie{Future sight}{Hazelnut beans, w/ steamed whole milk. Topped with milk foam}
            \item \displayRecepie{Dreamy Haze}{Hazelnut beans w/ steamed almond milk. Topped with milk foam}
            \item \displayRecepie{Crazy Capp}{Dark Roast beans w/ steamed skin milk. Topped with milk foam}
        \end{enumerate}

        \pagebreak
        \noindent \subtitle{Latte}
        
        \begin{enumerate}
            \item \displayRecepie{Brazilian Latte}{Brazilian beans, w/ steamed coconut milk $\times2$. Topped with Milk foam}
            \item \displayRecepie{Vanilla Latte}{French vanilla beans, w/ steamed whole milk $\times2$. Topped with milk foam}
            \item \displayRecepie{Hazel Latte}{Hazelnut beans, w/ steamed whole milk $\times2$. Topped with milk foam}
            \item \displayRecepie{Hazel Latte (no lactose)}{Hazelnut beans w/ steamed almond milk $\times2$. Topped with milk foam}
            \item \displayRecepie{Dark Roast Latte}{Dark Roast beans w/ steamed skin milk $\times2$. Topped with milk foam}
        \end{enumerate}
    \subsubsection{Baking}
    \subsubsection{Serving / Taking Orders}
    \subsubsection{Cleaning}
    \subsubsection{Ordering \& Storing}
    \subsubsection{Cooking}

    \linkSection{Player Data}

    \linkSection{Monetization}

    \linkSection{Map Design}

    \subsection{Main Menu Scenary Shot}

    \begin{spacing}{1.2}
        \noindent The Scenery shot of the main menu should be a table inside a building (possibly the cafe) that has a huge window looking towards the outisde
        the table should be big enough to fit 9 people 6 from the sides, 2 facing the window, and 1 looking from the windows onto the table,
        the table is suppose to have some stuff on it ie. a laptop, some graph papers, and possibly coffee cups. The outside of the window should be a road with road lamps,
        and possibly a forest behind the road, there should also be a \emph{swinging} lamp on the top as well. The camera will be looking at the table towards the window.
        For more detail please look at the reference image.
    \end{spacing}

    \begin{figure}[hb]
        \centering
        \includegraphics[width = .3\textwidth, keepaspectratio = true]{main_menu_ref.png}
        \caption{Reference image for main menu map}
        \label{Figure:Main Menu}
    \end{figure}

\end{document}