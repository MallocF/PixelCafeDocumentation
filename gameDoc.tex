\documentclass[a4paper, 10pt]{article}

\usepackage[margin = 1in]{geometry} % for spacing around
\usepackage{graphicx} % for including images in your pdfs
\usepackage{xcolor} % for including colors in your pdf
\usepackage{soul} % for text decoration
\usepackage[utf8]{inputenc} % for encoded text
\usepackage[T1]{fontenc}
\usepackage{setspace} % for setting different line spacings between paragrafs.
\usepackage{enumerate} % for letting us get more detailed enumerate lists
\usepackage{multirow} % to let us combine more rows together
\usepackage{colortbl} % for decorating tables
\usepackage{amsmath} % used for representing more complicated math displays
\usepackage{supertabular}
\usepackage{longtable} % both of these packages are used to making really big tables
\usepackage{wrapfig} % allows us to wrap text around figures
\usepackage{fancyhdr} % for making fancy headers
%\usepackage{bibtex} % for making better bibliographies
\usepackage[pdftex]{hyperref} % for letting us make links
\usepackage{lscape} % Allows us to flip from portrait to landspace
\usepackage{tikz} % for high detailed drawing
\usepackage{multicol} % To put things side by side
\usepackage{rotating} % For rotating objects
% \usepackage{draftwatermark} % For adding watermarks
\usepackage{MnSymbol} % for using multiple symbols
\usepackage{hyperref} % for hyper links

% Setting up the default image path
\graphicspath{{./Pictures/}}


% Setting up the fancy page style
\fancypagestyle{customStyle}{
    \lhead{} \chead{} \rhead{}
	\lfoot{} \cfoot{\thepage} \rfoot{}
	\renewcommand{\headrulewidth}{0pt}
	\renewcommand{\footrulewidth}{1pt}
}
\pagestyle{customStyle}
    
% Predefining custom colors
\definecolor{importantColor}{RGB}{255, 0, 0}
\definecolor{mediocerColor}{RGB}{255, 150, 37}
\definecolor{moneyColor}{RGB}{123, 230, 29}
\definecolor{noteColor}{RGB}{14, 192, 208}
\definecolor{commonRarityColor}{RGB}{191, 191, 191}
\definecolor{uncommonRarityColor}{RGB}{12, 186, 0}
\definecolor{rareRarityColor}{RGB}{0, 102, 204}
\definecolor{epicRarityColor}{RGB}{127, 0, 255}
\definecolor{legendaryRarityColor}{RGB}{255, 128, 0}
\definecolor{mythicalnRarityColor}{RGB}{204, 0, 102}
\definecolor{unusualItemQualityColor}{RGB}{212, 176, 32}

% Setting up hyperref options
\hypersetup {
    colorlinks = false,
    citecolor = black,
    filecolor = blue,
    linkcolor = blue,
    urlcolor = blue,
    pdftex
}

% Custom commands
\newcommand{\itemizeListDisplay}[1]{\sc{#1}}
\newcommand{\colorLink}[3]{\hyperlink{#2}{\textcolor{#1}{#3}}}
\newcommand{\linkSection}[1]{\hypertarget{Section:#1}{\section{#1}}}
\newcommand{\subtitle}[1]{\begin{large}{\textbf{#1}}\end{large}}
\newcommand{\specialProductSymbol}[0]{\begin{Large}\textcolor{yellow}{$\filledstar$}\end{Large}}
\newcommand{\common}{\textcolor{commonRarityColor}{\emph{Common}}}
\newcommand{\uncommon}{\textcolor{uncommonRarityColor}{\emph{Uncommon}}}
\newcommand{\rare}{\textcolor{rareRarityColor}{\emph{Rare}}}
\newcommand{\epic}{\textcolor{epicRarityColor}{\emph{Epic}}}
\newcommand{\legendary}{\textcolor{legendaryRarityColor}{\emph{Legendary}}}
\newcommand{\mythical}{\textcolor{mythicalnRarityColor}{\emph{Mythical}}}
\newcommand{\unusual}{\textcolor{unusualItemQualityColor}{\emph{Unusual}}}
\newcommand{\fakeCaption}[1]{
    \vspace{2mm}
    \noindent #1
}
\newcommand{\displayRecepie}[2]{
    \noindent #1:
    \begin{itemize}
        \item #2
    \end{itemize}
}
\newcommand{\displayItem}[2]{
    \noindent #1
    \begin{itemize}
        \item #2
    \end{itemize}
}
\newcommand{\displayProduct}[3]{
    \noindent \hyperlink{Product:#1}{#1} & \textcolor{moneyColor}{#2 Robux} & #3 \\ \hline
}
\newcommand{\linkProduct}[1]{\hypertarget{Product:#1}{\subsubsection{#1}}}
% Implementing the author detail
\title{Pixel Cafe Game Documentation}
\author{Aidan Abdulov}
\date{}

% Start work on the document
\begin{document}
	\maketitle
    \vspace{5mm}

	\begin{abstract}
	    \noindent This document was made for the soul reason to plan out the upcoming game "Pixel Cafe" on Roblox.
	    This document will mention everything from how the main menu should be designed, the npc interactions and logic
	    map plans and design, and anything else needed to help aid the development team to success.
    \end{abstract}
	\pagebreak

	\tableofcontents
	\pagebreak

    \linkSection{Main Menu}

    \subsection{Scenery shot}
    \noindent Upon a player joining after they are done loading their camera should be placed looking at an
    location of an \emph{custom build} map. Adding blurs and other visual effects is \textcolor{mediocerColor}{optional}.
    Their should be multiple versions of the map depending on the season, ie if it's winter a snowy version of the map
    should be loaded in, and if it's summer or some other season or an event like hallowen the map should be loaded to fit the mood.

    \vspace{2mm}
    \noindent For more info about the map design please click \colorLink{noteColor}{Section:Map Design}{here}
    \subsection{UI for the Main Menu}
    \noindent The next list will show all the options the main menu will have to have...

    \begin{itemize}
        \item \itemizeListDisplay{Party UI}
        \item \itemizeListDisplay{Store UI}
        \item \itemizeListDisplay{Trade UI}
        \item \itemizeListDisplay{Inventory UI}
        \item \itemizeListDisplay{Leaderboard / Rank UI}
        \label{list:MainMenuUI}
    \end{itemize}

    \subsubsection{Party UI}

    \noindent The party UI is a group of UI's that should give the player options to:
    \vspace{5mm}
    \begin{itemize}
        \item \itemizeListDisplay{Create parties}
        \item \itemizeListDisplay{Invite players to parties}
        \item \itemizeListDisplay{Remove players from parties}
        \item \itemizeListDisplay{Start a game with your party}
        \item \itemizeListDisplay{Join friend}
    \end{itemize}

    \vspace{5mm}
    \noindent When creating parties the party maker/creator should be allowed to make it an \textcolor{green}{opened} or
    \textcolor{red}{closed} party. If the party is open then it will be displayed for everyone to see and join, however if the party is closed then
    only the party creator can invite other players that \textcolor{importantColor}{aren't} in a party.
    \textcolor{importantColor}{Only the party creator} can kick / remove players from his party. And finally
    upon pressing play the game should teleport everyone in the party to the desired location, if the teleport fails
    the players are to be informed of the failed teleport. When creating a party the player will be presented with more then just one
    option to make the party opened or closed. The next list will show you all the options the player will be promted upon making a party.
    \vspace{2mm}

    \begin{itemize}
        \item \itemizeListDisplay{Party Type}
        \item \itemizeListDisplay{Friends can join session}
    \end{itemize}

    \vspace{3mm}
    \noindent If the party is an open type it will get a few more \textcolor{moneyColor}{bonus} option... such as
    \begin{itemize}
        \item \itemizeListDisplay{Minimum Level Requierment}
        \item \itemizeListDisplay{Minimum Days Survived}
    \end{itemize}

    \noindent Keep in mind that all of these bonus options are \textcolor{importantColor}{OPTIONAL}.
    \pagebreak

    \subsubsection{Store and Trade UI's}
    \noindent Before reading about these UI's it will come in \emph{handy} if you check out the \colorLink{noteColor}{Section:Player Data}{Player Data} section.
    \vspace{2mm}

    \noindent The store UI should allow the player to purchase the listed items:
    \vspace{3mm}

    \begin{itemize}
        \item \itemizeListDisplay{In game ``loot crates''}
        \item \itemizeListDisplay{Gamepass's}
        \item \itemizeListDisplay{Dev Products}
    \end{itemize}

    \vspace{2mm}
    \noindent To learn more about all of these please click \colorLink{noteColor}{Section:Monetization}{here}.\newline{}
    \noindent The Shop UI should offer the player options to select what they wanna buy, on the far right side it should
    display the item the player selected AND \emph{Show the items description}, it should also show you the items price,
    and the currency you would need to pay for the item.

    \vspace{2mm}
    \begin{spacing}{1.2}
        \noindent Once the player press's the \emph{trade button}, they will be promted with a player list that they can requests trades with.
        Upon sending a request if the player accept the request a new UI would be represented that will let the players put stuff from their inventory to the trading UI.
        The trading UI will be a giant frame, on the top of the frame will be a $6\times2$ grid, the first 6 slots will be for the first player and the other 6 for the other player.
        If they have multiple of the item they would be promoted with an option to select how much of \emph{that one item} should they put into the trade, this means they can have up to 6 \textcolor{importantColor}{\emph{unique}} items in the trade.
        On the right side of the main UI will be the other persons character frame with their name below them, and right below that would be a private chat between the two players.
        Below the trading part would be the players inventory with a filter bar on the top of that inventory frame, and right below the inventory will be an \textcolor{moneyColor}{accept} and \textcolor{red}{decline} button.    
    \end{spacing}

    \linkSection{Main Game}
    \noindent This section will talk about the main game mechanics.

    \subsection{Basic Gameplay Overview}
    \begin{spacing}{1.2}
        \noindent Once all the players get teleported and the game starts the players will need to wait for the
        party leader to chose a save file, once a save file has been selected and the entire cafe loaded the players
        will also get teleported into the cafe. If the game is brand new they get teleported into the cafe with all the default tools.
        The game will start off with a small storage unit, bakery, and the main cafe area, it will also start off with a small parking space
        which will be important to upgrade since the more parking space you got the more room for people to come you got. The game runs over two basic systems,
        money and ratings. If at any time your rating goes way below the \emph{requiered minimum rating} for staying open or if you do not manage to make
        enough money to pay daily taxes you will lose the game and the save file will be forcfully deleted. The game has a few different
        things that need to be managed by players, one of the things will be serving customers, a player is expected to go to a custom, talk to them and
        write everything down, upon writing everything down you will give the order to the barista who would take care of the order, and there are planty of other jobs
        that you can take a look at \hyperlink{Sub Section:Jobs}{\textcolor{noteColor}{here}}. Another aspect of the game has to do with the fact
        that food can and will get spoiled, and also ordering ingridients takes a lot of time, so making sure you order enough and order in things in time will be really important,
        upon the day finishing if the players haven't lost they will be promted with a continiue button so the players can have a breather or if they need to take a bathroom break.
        And that's the \emph{brief} overview of the game, for more detail on everything else please keep reading the document.
    \end{spacing}

    \hypertarget{Sub Section:Jobs}{\subsection{Jobs}}
        \subsubsection{Drinks Assembling}
            \noindent Drink assembling as the name says, is a job that requiers someone to assembly drinks, but before
            we can explain the process that has to be done, let's first off start listing everything we need for the drink assembler to do his/her job:
            \vspace{2mm}

            \begin{enumerate}
                \item Blender
                \item Grinding Machine
                \item Steamer
                \item Ice Maker
                \item Cup \& Lid Holder
                \item Syrup Holders / Dispensers
                \item Cream Holders / Dispensers
                \item Cutting Board
                \item Juicer
            \end{enumerate}

            \begin{center}
                \fakeCaption{\sc{List of tools}}
            \end{center}

            \pagebreak
            \begin{itemize}
                \item 
                    \noindent \itemizeListDisplay{Syrups}
                    \begin{enumerate}
                        \item Caramel
                        \item Chocolate
                        \item Vanilla
                        \item Honey
                    \end{enumerate}
                \item 
                    \noindent \itemizeListDisplay{Creamers}
                    \begin{enumerate}
                        \item Caramel
                        \item Chocolate
                        \item French Vanilla
                        \item Hazlenut
                        \item Coconut
                        \item Almond
                        \item Peppermint
                        \item Pumpkin Spice
                    \end{enumerate}
                \item 
                    \noindent \itemizeListDisplay{Milk}
                    \begin{enumerate}
                        \item Skin Milk
                        \item Whole Milk
                        \item Soy Milk
                        \item Almond Milk
                        \item Coconut Milk
                        \item Milk Foam
                    \end{enumerate}
                \item
                    \noindent \itemizeListDisplay{Toppings}
                    \begin{enumerate}
                        \item Cookies Crumbs
                        \item Chocolate Chips
                        \item Sprinkles
                        \item Cherrys
                        \item Lemons
                        \item Strawberries
                        \item Whipping Cream
                        \item 
                            \noindent Powders
                            \begin{enumerate}
                                \item Chocolate Powder
                                \item Vanilla Powder
                            \end{enumerate}
                    \end{enumerate}
                \item 
                    \noindent \itemizeListDisplay{Coffee Beans}
                    \begin{enumerate}
                        \item Brazilian Beans
                        \item French Vanilla Beans
                        \item Dark Roast Beans
                        \item Hazelnut Beans
                    \end{enumerate}
                \item
                    \noindent \itemizeListDisplay{Tea Leaves}
                    \begin{enumerate}
                        \item Black Tea
                        \item Green Tea
                        \item White Tea
                        \item Yellow Tea
                        \item Apple Tea
                        \item Turkish Tea
                        \item Raspberry Tea
                    \end{enumerate}
            \end{itemize}

            \begin{center}
                \fakeCaption{\sc{List of ingridients}}
            \end{center}
            \pagebreak

            \noindent \subtitle{Frappes}

            \begin{enumerate}
                \item \displayRecepie{ChocoMoco Frappe}{Hazlenut beans, w/ chocolate creamer, blended with ice. Then topped with whipping cream and chocolate syrup, finally, you will sprinkle some chocolate chips ontop of the whipping cream.}
                \item \displayRecepie{Caramel Blessing Frappe}{Steamed water, Brazilian beans, w/ caramel creamer, blended with ice. Then topped with whipping cream and Caramel syrup.}
                \item \displayRecepie{Vanilla Lovers Frappe}{Steamed water,  French vanilla roast beans, w/ French vanilla creamer, blended with Ice. Then topped with whipping cream and vanilla syrup, finally, you will sprinkle some vanilla powder ontop of the whipping cream.}
            \end{enumerate}

            \noindent \subtitle{Espresso}

            \begin{enumerate}
                \item \displayRecepie{Brazilian's Espresso}{Brazilian beans w/ steamed water and sugar}
                \item \displayRecepie{French Vanilla Espresso}{French vanilla beans w/ steamed water and sugar}
                \item \displayRecepie{Dark Roast Espresso}{Dark Roast beans w/ steamed water and sugar}
                \item \displayRecepie{Hazelnut Espresso}{Hazelnut beans w/ steamed water and sugar}
            \end{enumerate}

            \noindent \subtitle{Americano}
            
            \begin{enumerate}
                \item \displayRecepie{Brazilian's Espresso}{Brazilian beans w/ steamed water, more added water w/ sugar}
                \item \displayRecepie{French Vanilla Espresso}{French vanilla beans w/ steamed water, more added water w/sugar}
                \item \displayRecepie{Dark Roast Espresso}{Dark Roast beans w/ steamed water, more added water w/sugar}
                \item \displayRecepie{Hazelnut Espresso}{Hazelnut beans w/ steamed water, more added water w/sugar}
            \end{enumerate}

            \noindent \subtitle{Cappuchino}

            \begin{enumerate}
                \item \displayRecepie{Amazonian rain}{Brazilian beans, w/ steamed coconut milk. Topped with Milk foam}
                \item \displayRecepie{Vanilla perfection}{French vanilla beans, w/ steamed whole milk. Topped with milk foam}
                \item \displayRecepie{Future sight}{Hazelnut beans, w/ steamed whole milk. Topped with milk foam}
                \item \displayRecepie{Dreamy Haze}{Hazelnut beans w/ steamed almond milk. Topped with milk foam}
                \item \displayRecepie{Crazy Capp}{Dark Roast beans w/ steamed skin milk. Topped with milk foam}
            \end{enumerate}

            \pagebreak
            \noindent \subtitle{Latte}
            
            \begin{enumerate}
                \item \displayRecepie{Brazilian Latte}{Brazilian beans, w/ steamed coconut milk $\times2$. Topped with Milk foam}
                \item \displayRecepie{Vanilla Latte}{French vanilla beans, w/ steamed whole milk $\times2$. Topped with milk foam}
                \item \displayRecepie{Hazel Latte}{Hazelnut beans, w/ steamed whole milk $\times2$. Topped with milk foam}
                \item \displayRecepie{Hazel Latte (no lactose)}{Hazelnut beans w/ steamed almond milk $\times2$. Topped with milk foam}
                \item \displayRecepie{Dark Roast Latte}{Dark Roast beans w/ steamed skin milk $\times2$. Topped with milk foam}
            \end{enumerate}

            \noindent \subtitle{Black Coffee}

            \begin{enumerate}
                \item \displayRecepie{Brazilian Coffee}{Brazilian beans w/ steamed water}
                \item \displayRecepie{French Vanilla Coffee}{French vanilla beans w/ steamed water}
                \item \displayRecepie{Dark Roast Coffee}{Dark Roast beans w/ steamed water}
                \item \displayRecepie{Hazelnut Coffee}{Hazelnut beans w/ steamed water}
            \end{enumerate}

            \noindent \subtitle{Mocha}

            \begin{enumerate}
                \item \displayRecepie{Vanilla Mocha}{Steamed water, French vanilla beans w/ chocolate creamer. Topped with whipped cream, vanilla syrup, and chocolate powder}
                \item \displayRecepie{Crazy Mocha}{Steamed water, Dark Roast beans w/ chocolate creamer. Topped with whipped cream, and chocolate syrup}
                \item \displayRecepie{True Mocha}{Steamed water, Hazelnut beans w/ chocolate creamer. Topped with whipped cream, Chocolate syrup, Chocolate powder, Chocolate chips}
            \end{enumerate}
            \subsubsection{Baking}
                \noindent This job is one of the bigger ones, as someone who would do baking the cafe depends a lot on you. This job requiers
                you to make delicious pastries, cakes, muffins... etc. Make sure to read about food quality \hyperlink{Sub Section:Food System}{\textcolor{noteColor}{here}}, as this job highly depends on it.
                Some tools you will be using for this job are as follow:

                \begin{itemize}
                    \item Ovens
                    \item Conventional Ovens
                    \item Deep Fryer
                    \item Cookie Tray
                    \item Biscuit Tray 
                    \item Muffin Tray
                    \item Donut Tray
                    \item Pie Tray
                    \item Cake Pan
                    \item Mixing Bowl 
                \end{itemize}

                \vspace{2mm}
                \noindent And here is a list of all the thing you can make with your equipment.

                \begin{itemize}
                    \item Baking:
                        \begin{enumerate}
                            \item Croissants
                            \item Chocolate Croissant
                            \item Classic Buttered Croissant
                            \item Strawberry Croissant
                        \end{enumerate}
                    \item Pies:
                        \begin{enumerate}
                            \item Cherry Pie
                            \item Strawberry Pie
                            \item Blueberry Pie  
                        \end{enumerate}
                    \item Cakes:
                        \begin{enumerate}
                            \item Mini Chocolate Cake
                            \item Mini Vanilla Cake
                            \item Mini Strawberry Cake
                        \end{enumerate}
                    \item Muffins:
                        \begin{enumerate}
                            \item Blueberry Muffin
                            \item Strawberry Muffin
                            \item Icing Muffin  
                        \end{enumerate}
                    \item Donuts:
                        \begin{enumerate}
                            \item Strawberry Icing Donut
                            \item Plain Suger Donut
                            \item Chocolate Icing Donut
                            \item Vanilla Icing Donut                           
                        \end{enumerate}
                    \item Biscuits:
                        \begin{enumerate}
                            \item Buttered Biscuit
                        \end{enumerate}
                    \item Rolls:
                        \begin{enumerate}
                            \item Cinnamon Roll
                            \item Sweet Roll                             
                        \end{enumerate}
                    \item Cookies:
                        \begin{enumerate}
                            \item Chocolate Cookie
                            \item Plain Cookie
                            \item Raisan Cookie
                        \end{enumerate}
                \end{itemize}

                \vspace{2mm}
                \begin{itemize}
                    \item Baking:
                        \begin{itemize}
                            \item To get croissant batter; get a mixing bowl, put 1 egg, 1/2 stick of butter, 1/4th carton of milk, and 1/2 bag of flour, and then 1/4th container of yeast.
                        \end{itemize}
                        \begin{enumerate}
                            \item \displayRecepie{Chocolate Croissant}{get the batter and mix it with chocolate chips, then put it on a cookie tray and put it in the conventional oven}
                            \item \displayRecepie{Classic Buttered Croissant}{get the batter and put it in the conventional oven, then put 1/8th stick of butter on the top of the croissants}
                            \item \displayRecepie{Strawberry Croissant}{get the batter and mix it with blended strawberrys, then put it on a cookie tray and put it in the conventional oven}
                            \item \displayRecepie{Pies}{To get cake batter; get a mixing bowl, 2 eggs, 1/2 bag of baking soda, 1/2 carton of milk, 1/2 bag of flour.}
                            \item \displayRecepie{Cherry Pie}{Get the batter and put it in a pie tray, then put blended cherrys in the pie and put it in the oven}
                            \item \displayRecepie{Strawberry Pie}{Get the batter and put it in a pie tray, then put blended strawberrys in the pie and put it in the oven}
                            \item \displayRecepie{Blueberry Pie}{get the batter and put it in the pie tray, then put blended blueberrys in the pie and put it in the oven}    
                        \end{enumerate}
                    \item Cakes:
                        \begin{itemize}
                            \item To get cake batter; get a mixing bowl, and input 1/2 a stick of butter, 1/2 a bag of sugar, 2 eggs, 1/2 bag of baking soda, 1/2 carton of milk, 1/2 bag of flour.
                        \end{itemize}
                        \begin{enumerate}
                            \item \displayRecepie{Mini Chocolate Cake}{Get your cake batter, input chocolate chips inside the cake batter when you have it in the mixing bowl and input it into a cake tray, put it in the oven, and then take it out and put chocolate Icing ontop.}
                            \item \displayRecepie{Mini Vanilla Cake}{Get your cake batter and input it into a cake tray, and put it in the oven, then take it out and put vanilla icing on it.}
                            \item \displayRecepie{Mini Strawberry Cake}{Get your cake batter and input it into a cake tray, and put it in th oven, then take it out and put strawberry icing on it, then put chopped strawberries ontop of the cake}
                        \end{enumerate}

                    \item Muffins:
                        \begin{itemize}
                            \item To get muffin batter; get a mixing bowl, put 1/2 a stick of butter, 1/4 a bag of sugar, 1 egg, 1/4 bag of baking soda, 1/4 carton of milk, and 1/4 bag of flour.
                        \end{itemize}
                        \begin{enumerate}
                            \item \displayRecepie{Blueberry Muffin}{Get your muffin batter, and put blueberrys in the muffin batter, and input it into a muffin tray, take it out and you're good}
                            \item \displayRecepie{Strawberry Muffin}{Get your muffin batter, and put chopped strawberries in the muffin batter, and input it into a muffin tray, take it out and you're good}
                            \item \displayRecepie{Icing Muffin}{Get your muffin batter and put it into a muffin tray, then put it in the oven and and when you take it out, put vanilla icing ontop of the muffin.}
                        \end{enumerate}
                    \item Donuts:
                        \begin{itemize}
                            \item To get donut batter; get a mixing bowl, put 1/4 a stick of butter, 1/4 a bag of sugar, 1 egg, 1/4 bag of baking soda, 1/4 carton of milk, and 1/4 bag of flour.
                        \end{itemize}

                        \begin{enumerate}
                            \item \displayRecepie{Strawberry Icing Donut}{Get your donut batter, put it in a donut tray, and wait for it to harden, then take the hardened donut and put it in the deep frier, take the donut out and put strawberry icing and sprinkles on the donut}
                            \item \displayRecepie{Plain Suger Donut}{Get your donut batter, put it in a donut tray, and wait for it to harden, then take the hardened donut and put it in the deep frier, take the donut out and put sugar on the donut}
                            \item \displayRecepie{Chocolate Icing Donut}{Get your donut batter, put it in a donut tray, and wait for it to harden, then take the hardened donut and put it in the deep frier, take the donut out and put chocolate syrup and chocolate chips on the donut}
                            \item \displayRecepie{Vanilla Icing Donut}{Get your donut batter, put it in a donut tray, and wait for it to harden, then take the hardened donut and put it in the deep frier, take the donut out and put vanilla icing and sprinkles on the donut}
                        \end{enumerate}

                    \item Biscuits: 
                        \begin{itemize}
                            \item To get biscuit batter; get a mixing bowl, and put 1/4th stick of butter, 1/4th baking soda, 1 egg, 1/4th flour, 1/4th carton of milk
                        \end{itemize}

                        \begin{enumerate}
                            \item \displayRecepie{Buttered Biscuit}{get your biscuit batter, put it in the biscuit tray, put it in the oven, and then when its done, put 1/8th stick of butter on the biscuit}
                        \end{enumerate}
                    \item Rolls:
                        \begin{itemize}
                            \item To get Roll batter; get a mixing bowl, put 1/4th stick of butter, 1/4th bag of baking soda, 1 egg, 1/4th flour, 1/2 carton of milk, and 1/4th bag of sugar
                        \end{itemize}

                        \begin{enumerate}
                            \item \displayRecepie{Cinnamon Roll}{get the roll batter and put them in a cookie tray in the oven, take them out and then put vanilla icing, and cinnamon powder ontop}
                            \item \displayRecepie{Sweet Roll}{get the roll batter, and put them in a cookie tray in he oven, take them out and put chocolate icing and chocolate chips on the top of the roll}
                        \end{enumerate}
                    \item Cookies:
                        \begin{itemize}
                            \item To get cookie batter; get a mixing bowl, put 1/4th stick of butter, 1/4th bag of baking soda, 1 egg, 1/4th flour, 1/4th carton of milk, and 1/2 bag of sugar
                        \end{itemize}

                        \begin{enumerate}
                            \item \displayRecepie{Chocolate Chip Cookie}{Get the cookie batter and mix chocolate chips in, put it on a cookie tray and it in the oven}
                            \item \displayRecepie{Plain Cookie}{Get the cookie batter, and put it in a cookie tray, then put it in the oven}
                            \item \displayRecepie{Raisan Cookie}{Get the cookie batter and mix raisans with the batter, and put it in a cookie tray, then put it in the oven}
                        \end{enumerate}
                \end{itemize}
            \subsubsection{Serving / Taking Orders}
                \noindent This job is self explanatory, you would be going around the cafe and taking orders from the npcs, you will have to write the orders down
                and be sure to be very descriptive because your baker / barista will highly depend on your note taking skills, and further more
                you will need to make it very clear for what table the order is for so you can bring the food / drinks to the proper table.
                Some tools that you will be using are as follow:

                \begin{enumerate}
                    \item Note Book
                    \item Tray
                    \item Cash Register
                \end{enumerate}
            \subsubsection{Cleaning}
                \noindent During the day there will be a lot of mess that needs to be cleaned in order to keep the cafes good reputation!
                For example maybe the baristas would need more clean cups, and maybe the waiters would need more clean plates to put the bakery products on them, it's
                the cleaners job to keep everything clean and ready! Here are the thing that players would be using if they decided to clean:

                \begin{enumerate}
                    \item Janitor Cart (This is basically a lot of different things combines)
                    \item Cleaning Sprays
                    \item Sink
                    \item Broom
                \end{enumerate}
            \subsubsection{Ordering \& Storing}
                \noindent During the day your co workers and you will need suplies to do certain things... obviously you can't
                do cooking or drink assembling if you don't have the right ingridients or even tools for the job. Be sure to know when ordering
                you can \textcolor{importantColor}{\emph{ONLY MAKE ONE ORDER AT A TIME}}. Once an order is made you will need to wait for the delivery truck to arrive
                when the delivery truck arrives you will also need to empty it and drag everything in the proper storage unit before it goes bad. Be sure u get everything out because
                you wont be able to make another order until the truck if emptied out and on its way. Some tools that you will use for your job go as follow:

                \begin{enumerate}
                    \item Ordering Tablet
                    \item Platform Trolley Cart
                    \item Thermostat
                \end{enumerate}
            \subsubsection{Cooking}
                \noindent Cooking is an important job for the cafe as a lot of the npc's will be asking for snacks with their drinks.
                Cooking consists of food that can lose quality really fast so this job will requier a lot of planning and smart preperations beforehand.
                \vspace{2mm}

                \noindent Some of the things you will be requiered to cook are:
                \begin{itemize}
                    \item 
                        \noindent Sandwiches
                        \begin{enumerate}
                            \item Egg Salad Sandwich
                            \item Turkey \& Cheese Sandwich
                            \item Chicken \& Cheese Sandwich
                            \item Grilled Cheese Sandwich
                            \item Ham \& Cheese Sandwich
                        \end{enumerate}
                    \item 
                        \noindent Eggs
                        \begin{enumerate}
                            \item Omelet
                            \item Sunny-Side-Up
                            \item Scrambled eggs
                        \end{enumerate}
                    \item 
                        \noindent Soups
                        \begin{enumerate}
                            \item Tomato Soup
                            \item Chicken Noodle Soup
                        \end{enumerate}
                    \item 
                        \noindent Salads
                        \begin{enumerate}
                            \item Chicken Salad
                            \item Seasonal Salad
                            \item Caesar Salad
                            \item Greek Salad
                            \item Classical Salad
                        \end{enumerate}
                    \item 
                        \noindent Pancakes
                        \begin{enumerate}
                            \item Blueberry Pancakes
                            \item Plain Pancakes
                            \item Strawberry Pancakes
                        \end{enumerate}
                    \item 
                        \noindent Waffles
                        \begin{enumerate}
                            \item Plain Waffles
                            \item Blueberry Waffles
                            \item Strawberry Waffles
                        \end{enumerate}
                    \item 
                        \noindent Toast
                        \begin{enumerate}
                            \item Avocado Toast
                            \item Buttered Toast 
                        \end{enumerate}
                \end{itemize}

                \vspace{4mm}
                \noindent The tools you will be using to make all of these things are as follow:

                \begin{enumerate}
                    \item Oven
                    \item Stove Tops
                    \item Toaster
                    \item Cutting Board
                    \item Blender
                    \item Sandwich Prepping Station
                    \item Salad Prepping Station 
                \end{enumerate}

                \vspace{4mm}
                \subtitle{Recepies}

                \begin{itemize}
                    \item 
                        \noindent Sandwiches
                        \begin{enumerate}
                            \item \displayRecepie{Egg Salad Sandwich}{Cooked scrambled egg, with lettuce, mayo, and toasted whole grain bread.}
                            \item \displayRecepie{Turkey \& Cheese Sandwich}{Turkey, with cheese, with whole grain bread.}
                            \item \displayRecepie{Chicken \& Cheese Sandwich}{Chicken, with cheese, with whole grain bread.}
                            \item \displayRecepie{Grilled Cheese Sandwich}{Cheese, in-between whole grain bread, then brought over to the stove top, cooked on each side.}
                            \item \displayRecepie{Ham \& Cheese Sandwich}{Ham, with cheese, with whole grain bread.}
                        \end{enumerate}
                    \item 
                        \noindent Eggs
                        \begin{enumerate}
                            \item \displayRecepie{Omelet}{Egg, with cheese, lettuce, pepper, and onion, cooked in a frying pan ontop of the stove top.}
                            \item \displayRecepie{Sunny Side Up}{Egg that is only cooked on 1 side.}
                            \item \displayRecepie{Scrambled eggs}{Egg that is scrambled in the pan.}
                        \end{enumerate}
                    \item 
                        \noindent Waffles
                        \begin{enumerate}
                            \item \displayRecepie{Plain Waffle}{Egg, Flour, and Milk blended, then later added to the waffle maker}
                            \item \displayRecepie{Blueberry Waffles}{Egg, Flour, Milk, and Blueberrys blended, then later added to the waffle maker}
                            \item \displayRecepie{Strawberry Waffles}{Egg, Flour, Milk, and Strawberrys blended, then later added to the waffle maker}
                        \end{enumerate}
                    \item 
                        \noindent Toast
                        \begin{enumerate}
                            \item \displayRecepie{Avocado Toast}{Toast whole grain bread, then topped with avacado}
                            \item \displayRecepie{Buttered Toast}{Toast Whole grain bread, then topped with butter}
                        \end{enumerate}
                    \item 
                        \noindent Soups
                        \begin{enumerate}
                            \item \displayRecepie{Tomato Soup}{Blended tomatos, with milk, set to a boil in a pot.}
                            \item \displayRecepie{Chicken Noodle Soup}{Cook noodles, with chicken added to it in a pot.}
                        \end{enumerate}
                    \item 
                        \noindent Pancakes
                        \begin{enumerate}
                            \item \displayRecepie{Blueberry Pancakes}{Egg, flour, milk, and blueberrys blended, then later added to a frying pan.}
                            \item \displayRecepie{Plain Pancakes}{Egg, Flour, and Milk blended, then later added to a frying pan}
                            \item \displayRecepie{Strawberry Pancakes}{Egg, Flour, Milk, and strawberrys blended, then later added to a frying pan}
                        \end{enumerate}
                    \item 
                        \noindent Salads
                        \begin{enumerate}
                            \item \displayRecepie{Caesar Salad}{Lettuce, Croutons, Olive Oil, Lemon Juice, and Cheese}
                            \item \displayRecepie{Chicken Salad}{Chicken, Lettuce, tomato, mayonnaise, grounded black pepper.}
                            \item \displayRecepie{Greek Salad}{Tomatos, cucumbers, olives, feta cheese, and onions. And dressed with olive oil}
                            \item \displayRecepie{Classical Salad}{Lettuce, Tomato, onions, peppers, cheese, and ranch dressing}
                            \item \displayRecepie{Seasonal Salad}{Lettuce, Tomato, Ham, \& Cheese Salad. dressed with olive oil}
                        \end{enumerate}
                \end{itemize}

                \vspace{4mm}
                \subtitle{Ingridients}

                \begin{itemize}
                    \item
                        \noindent Meat:
                        \begin{enumerate}
                            \item Chicken
                            \item Turkey
                            \item Ham
                            \item Egg
                        \end{enumerate}
                    \item
                        \noindent Grains:
                        \begin{enumerate}
                            \item Whole Grain Bread
                            \item Croutons
                            \item Flour
                        \end{enumerate}
                    \item
                        \noindent Veggies:
                        \begin{enumerate}
                            \item Avacado
                            \item Lettuce
                            \item Onion
                            \item Tomato
                            \item Cucumbers
                            \item Olives
                            \item Peppers
                        \end{enumerate}
                    \item
                        \noindent Berries \& Fruits:
                        \begin{enumerate}
                            \item BlueBerries
                            \item Strawberries
                        \end{enumerate}
                    \item
                        \noindent Cheese:
                        \begin{enumerate}
                            \item Generic Cheese
                            \item Feta Cheese
                        \end{enumerate}
                    \item
                        \noindent Dressings:
                        \begin{enumerate}
                            \item Lemon Juice
                            \item Olive Oil
                            \item Ranch Dressing
                        \end{enumerate}
                    \item
                        \noindent Others:
                        \begin{enumerate}
                            \item Noodles/pasta
                            \item Grounded Black Pepper
                            \item Salt
                            \item Mayonnaise
                            \item Ketchup
                        \end{enumerate}
                \end{itemize}
        
        \hypertarget{Sub Section:Food System}{\subsection{Food System}}
                
        \linkSection{Tools}

        \subsection{Player Interactions \& Item Basics}
            \begin{spacing}{1.2}
                \noindent When a player hovers over an interectable item the item will start to glow to indicate that the item can be interected with. There
                are three types of interectable items:
                \begin{enumerate}
                    \item Pickable Items
                    \item Usable Items
                    \item Pickable \& Usable Items
                \end{enumerate}

                \noindent Pickable Items are items that can only be picked up and can not be used directly (ie any ingridient), Usable items are items that can \emph{only be used} and
                \textcolor{importantColor}{\emph{not}} picked up (ie sink), and pickable \& usable items are items that can be both picked up and then used after being picked up (ie knife).

                \vspace{3mm}
                \noindent An example of using all of these 3 would be as follows, you would pick up an lemon (pickable item) and drop it off at the cutting board (usable item), afterwards you would
                take a knife (pickable \& usable item) and use it on the cutting board to cut the lemon.

                \vspace{2mm}
                \noindent If you were to press \textcolor{noteColor}{\emph{E}} while hovering over an item
                three things can happen... First if you are empty handed you will pick up the item \textbf{IF}
                \textcolor{importantColor}{\emph{the item is pickable item type}}, second thing that can happen is if you
                already have an item in your hands, if this item is an \emph{pickable item} then nothing will happen, however if it's
                the item the player is holding is \emph{pickable \& usable item} \textbf{AND} if an interection between them exists then the given
                interection will happen, if not then nothing will happen.

                \vspace{2mm}
                \noindent If you were to press \textcolor{noteColor}{\emph{F}} then you will drop the item you are currently holding,
                if you are not holding anything in your hands nothing will happen.
            \end{spacing}
            \vspace{2mm}

            \subsection{Pickable Items}
                \begin{enumerate}
                    \item Ice
                    \item Fruits
                    \item \displayItem{Supply Bags}{You get them from ordering specific ingridients, they usually contain multiple of an ingridient type in them}
                    \item \displayItem{Dishes}{Used for putting different types of food on it}
                \end{enumerate}
            \subsection{Usable Items}
                \begin{enumerate}
                    \item \displayItem{Blender}{This machine is used for blending different fruits into smoothies, and it's also used for blending ice to make cold drinks}
                    \item \displayItem{Grinding Machine}{This Machine is used for grinding coffee beans}
                    \item \displayItem{Steamer}{This machine is used for warming up drinks or ingridients such as milk and or water}
                    \item \displayItem{Ice Maker}{This machine will convert water into ice overtime}
                    \item \displayItem{Cup \& Lid Holder}{This is just a place to store your cups}
                    \item \displayItem{Syrup Holders / Dispensers}{As the name says this machine dispenses different syrups}
                    \item \displayItem{Cream Holders / Dispensers}{As the name says this machine dispenses different creams}
                    \item \displayItem{Cutting Board}{This tool is used for cutting different fruits into smaller parts}
                    \item \displayItem{Juicer}{This machine is used for squishing fruit halfs into the glass}
                    \item \displayItem{Grater}{This is used to peel fruits into smaller parts}
                    \item \displayItem{Sink}{Used as a water supply, also used to clean dishes}
                    \item \displayItem{Trash}{Used for item disposal}
                \end{enumerate}
            \subsection{Pickable \& Usable Items}
                \begin{enumerate}
                    \item \displayItem{Knife}{Combined with cutting board this tool can cut fruits into halfs and halfs into pieces}
                    \item \displayItem{Tea Bags}{When combined with a cup with boiling water it can be used to make tea}
                    \item \displayItem{Scooper}{When used on supply bag, you will take some of the supplies out of the bag}
                    \item \displayItem{Bucket}{Used to carry things like water, ice, blended ice... etc}
                \end{enumerate}
    
                \linkSection{Player Data}

    \linkSection{Monetization}
                \subsection{In Game Currency}
                    \begin{spacing}{1.2}
                        \noindent During the main game you will be able to work different pletra of jobs, while doing these jobs
                        your payment behind the scenes will increase, if you for example did 90\% of the drinks, and 20\% of the pastries
                        you will get payed for the amount of work you did. During the main game there are \emph{two} sets of money pools. First one
                        is the cafes bank account which is the money that you can only use for buying stuff for the cafe, and the second one is
                        your own money which you can \emph{only} use on your self to buy stuff from the in game store.
                    \end{spacing}
                
                \subsection{Dev Products}
                    \vspace{3mm}
                    \begin{center}
                        \begin{tabular}{|c|c|c|}
                            \hline
                            \sc{Dev Product Name} & \sc{Price} & \sc{Dev Product ID} \\ \hline
                            \displayProduct{+\$10}{10 Robux}{N.A}
                            \displayProduct{+\$100}{90 Robux}{N.A}
                            \displayProduct{+\$500}{400 Robux}{N.A}
                            \displayProduct{+\$1000}{720 Robux}{N.A}
                        \end{tabular}
                    \end{center}
                    
                    \linkProduct{+\$10}
                        \noindent Gives you +10\$
                    \linkProduct{+\$100}
                        \noindent Gives you +100\$
                    \linkProduct{+\$500}
                        \noindent Gives you +500\$
                    \linkProduct{+\$1000}
                        \noindent Gives you +1000\$

                \subsection{Game Pass's}
                    \vspace{3mm}
                    \begin{center}
                        \begin{tabular}{|c|c|c|}
                            \hline
                            \sc{Gamepass Name} & \sc{Price} & \sc{Dev Product ID} \\ \hline
                            \displayProduct{VIP}{1250}{N.A}
                        \end{tabular}
                    \end{center}

                    \linkProduct{VIP}
                        \noindent \textcolor{yellow}{VIP} will provide a player with
                        \begin{itemize}
                            \item[\specialProductSymbol] \itemizeListDisplay{$2\times$ token gain}
                            \item[\specialProductSymbol] \itemizeListDisplay{$1.5\times$ movement speed}
                            \item[\specialProductSymbol] \itemizeListDisplay{$1.2\times$ coin gain}
                        \end{itemize}
                    
                \hypertarget{SubSection: Tokens}{\subsection{Tokens}}
                        \vspace{3mm}
                        \noindent Tokens are an in game currency that can only be obtained via daily rewards, or being in the lobby for an hour.
                        The tokens are a currecny that can \textcolor{importantColor}{\emph{not}} be bought.
                        
                        \noindent The daily rewards go as follow:
                        \begin{description}
                            \item[Day 1] 5 Tokens
                            \item[Day 2] 7 Tokens
                            \item[Day 3] 10 Tokens
                            \item[Day 4] 15 Tokens
                            \item[Day 5 (End)] 25 Tokens
                        \end{description}

                        \vspace{2mm}
                        \noindent Things that can effect your token gain are:
                        \begin{description}
                            \item[\hyperlink{Product:VIP}{VIP}] Gives $2\times$ the tokens.
                            \item[Premium] Gives 3 Tokens instead of 1 per hour.
                        \end{description}
                
                \subsection{Loot Crates}
                        \vspace{3mm}
                        \noindent Loot crates are items that can be bought off the in game store for currency or gotten via \hyperlink{SubSection: Tokens}{\textcolor{noteColor}{tokens}}.
                        Loot crates come in a pletra of rarities, and each rarity brings with it self a pletra of different \% for all the rarities.
                        \vspace{2mm}
                        
                        \noindent Rarities go as follow:
                        \begin{itemize}
                            \item \common
                            \item \uncommon
                            \item \rare
                            \item \epic
                            \item \legendary
                            \item \mythical
                        \end{itemize}

                        \noindent Loot crates drop clothing accesories that you can wear on your character, there accesorier can also be \unusual.
                        Unusual accesoriers will just come with added particle effects, particle effects vary and are completly random. Apart from accesories
                        you can also get taunts which also have their own rarities and can also be \unusual.

                        \vspace{2mm}
                        \noindent The following list shows all the boxes and their rarity drop chance:
                        \begin{itemize}
                            \item 
                                \noindent \common~Crates:
                                \begin{itemize}
                                    \item[$\bullet$] \common~$\longmapsto 65\%$ 
                                    \item[$\bullet$] \uncommon~$\longmapsto 30\%$ 
                                    \item[$\bullet$] \rare~$\longmapsto 4.5\%$ 
                                    \item[$\bullet$] \epic~$\longmapsto 0.5\%$ 
                                \end{itemize}
                            \item
                                \noindent \rare~Crates:
                                \begin{itemize}
                                    \item[$\bullet$]~\common $\longmapsto 37\%$ 
                                    \item[$\bullet$]~\uncommon $\longmapsto 40\%$ 
                                    \item[$\bullet$]~\rare $\longmapsto 20\%$ 
                                    \item[$\bullet$]~\epic $\longmapsto 2.9\%$ 
                                    \item[$\bullet$]~\legendary $\longmapsto 0.1\%$ 
                                \end{itemize}
                            \item
                                \noindent \epic~Crates:
                                \begin{itemize}
                                    \item[$\bullet$]~\uncommon $\longmapsto 50\%$ 
                                    \item[$\bullet$]~\rare $\longmapsto 40\%$ 
                                    \item[$\bullet$]~\epic $\longmapsto 8\%$ 
                                    \item[$\bullet$]~\legendary $\longmapsto 1.99\%$ 
                                    \item[$\bullet$]~\mythical $\longmapsto 0.01\%$ 
                                \end{itemize}
                            \item
                                \noindent \textcolor{red}{\emph{Hyper}} Crates:
                                \begin{itemize}
                                    \item[$\bullet$]~\rare $\longmapsto 60\%$ 
                                    \item[$\bullet$]~\epic $\longmapsto 36\%$
                                    \item[$\bullet$]~\legendary $\longmapsto 3.9\%$  
                                    \item[$\bullet$]~\mythical $\longmapsto 0.1\%$  
                                \end{itemize}
                        \end{itemize}
    \linkSection{Map Design}

    \subsection{Main Menu Scenary Shot}

    \begin{spacing}{1.2}
        \noindent The Scenery shot of the main menu should be a table inside a building (possibly the cafe) that has a huge window looking towards the outisde
        the table should be big enough to fit 9 people 6 from the sides, 2 facing the window, and 1 looking from the windows onto the table,
        the table is suppose to have some stuff on it ie. a laptop, some graph papers, and possibly coffee cups. The outside of the window should be a road with road lamps,
        and possibly a forest behind the road, there should also be a \emph{swinging} lamp on the top as well. The camera will be looking at the table towards the window.
        For more detail please look at the reference image.
    \end{spacing}

    \begin{figure}[hb]
        \centering
        \includegraphics[width = .3\textwidth, keepaspectratio = true]{main_menu_ref.png}
        \caption{Reference image for main menu map}
        \label{Figure:Main Menu}
    \end{figure}

\end{document}