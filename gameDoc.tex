\documentclass[a4paper, 10pt]{article}

\usepackage[margin = 1in]{geometry} % for spacing around
\usepackage{graphicx} % for including images in your pdfs
\usepackage{xcolor} % for including colors in your pdf
\usepackage{soul} % for text decoration
\usepackage[utf8]{inputenc} % for encoded text
\usepackage[T1]{fontenc}
\usepackage{setspace} % for setting different line spacings between paragrafs.
\usepackage{enumerate} % for letting us get more detailed enumerate lists
\usepackage{multirow} % to let us combine more rows together
\usepackage{colortbl} % for decorating tables
\usepackage{amsmath} % used for representing more complicated math displays
\usepackage{supertabular}
\usepackage{longtable} % both of these packages are used to making really big tables
\usepackage{wrapfig} % allows us to wrap text around figures
\usepackage{fancyhdr} % for making fancy headers
%\usepackage{bibtex} % for making better bibliographies
\usepackage[pdftex]{hyperref} % for letting us make links
\usepackage{lscape} % Allows us to flip from portrait to landspace
\usepackage{tikz} % for high detailed drawing
\usepackage{multicol} % To put things side by side
\usepackage{rotating} % For rotating objects
\usepackage{hyperref} % for hyper links

% Setting up the default image path
\graphicspath{./Pictures}


% Setting up the fancy page style
\fancypagestyle{customStyle}{
    \lhead{} \chead{} \rhead{}
	\lfoot{} \cfoot{\thepage} \rfoot{}
	\renewcommand{\headrulewidth}{0pt}
	\renewcommand{\footrulewidth}{1pt}
}
\pagestyle{customStyle}
    
% Predefining custom colors
\definecolor{importantColor}{RGB}{255, 0, 0}
\definecolor{mediocerColor}{RGB}{255, 150, 37}
\definecolor{moneyColor}{RGB}{123, 230, 29}
\definecolor{noteColor}{RGB}{14, 192, 208}
    
% Setting up hyperref options
\hypersetup {
    colorlinks = false,
    citecolor = black,
    filecolor = blue,
    linkcolor = blue,
    urlcolor = blue,
    pdftex
}

% Custom commands
\newcommand{\itemizeListDisplay}[1]{\sc{#1}}
\newcommand{\colorLink}[3]{\hyperlink{#2}{\textcolor{#1}{#3}}}
\newcommand{\linkSection}[1]{\hypertarget{Section:#1}{\section{#1}}}

% Implementing the author detail
\title{Pixel Cafe Game Documentation}
\author{Aidan Abdulov}
\date{}

% Start work on the document
\begin{document}
	\maketitle
    \vspace{5mm}

	\begin{abstract}
	    \noindent This document was made for the soul reason to plan out the upcoming game "Pixel Cafe" on Roblox.
	    This document will mention everything from how the main menu should be designed, the npc interactions and logic
	    map plans and design, and anything else needed to help aid the development team to success.
    \end{abstract}
	\pagebreak

	\tableofcontents
	\pagebreak

    \linkSection{Main Menu}

    \subsection{Scenery shot}
    \noindent Upon a player joining after they are done loading their camera should be placed looking at an
    location of an \emph{custom build} map. Adding blurs and other visual effects is \textcolor{mediocerColor}{optional}.
    Their should be multiple versions of the map depending on the season, ie if it's winter a snowy version of the map
    should be loaded in, and if it's summer or some other season or an event like hallowen the map should be loaded to fit the mood.

    \subsection{UI for the Main Menu}
    \noindent The next list will show all the options the main menu will have to have...

    \begin{itemize}
        \item \itemizeListDisplay{Party UI}
        \item \itemizeListDisplay{Store UI}
        \item \itemizeListDisplay{Trade UI}
        \item \itemizeListDisplay{Inventory UI}
        \item \itemizeListDisplay{Leaderboard / Rank UI}
        \label{list:MainMenuUI}
    \end{itemize}

    \subsubsection{Party UI}

    \noindent The party UI is a group of UI's that should give the player options to:
    \vspace{5mm}
    \begin{itemize}
        \item \itemizeListDisplay{Create parties}
        \item \itemizeListDisplay{Invite players to parties}
        \item \itemizeListDisplay{Remove players from parties}
        \item \itemizeListDisplay{Start a game with your party}
        \item \itemizeListDisplay{Join friend}
    \end{itemize}

    \vspace{5mm}
    \noindent When creating parties the party maker/creator should be allowed to make it an \textcolor{green}{opened} or
    \textcolor{red}{closed} party. If the party is open then it will be displayed for everyone to see and join, however if the party is closed then
    only the party creator can invite other players that \textcolor{importantColor}{aren't} in a party.
    \textcolor{importantColor}{Only the party creator} can kick / remove players from his party. And finally
    upon pressing play the game should teleport everyone in the party to the desired location, if the teleport fails
    the players are to be informed of the failed teleport. When creating a party the player will be presented with more then just one
    option to make the party opened or closed. The next list will show you all the options the player will be promted upon making a party.
    \vspace{2mm}

    \begin{itemize}
        \item \itemizeListDisplay{Party Type}
        \item \itemizeListDisplay{Friends can join session}
    \end{itemize}

    \vspace{3mm}
    \noindent If the party is an open type it will get a few more \textcolor{moneyColor}{bonus} option... such as
    \begin{itemize}
        \item \itemizeListDisplay{Minimum Level Requierment}
        \item \itemizeListDisplay{Minimum Days Survived}
    \end{itemize}

    \noindent Keep in mind that all of these bonus options are \textcolor{importantColor}{OPTIONAL}.
    \pagebreak

    \subsubsection{Store and Trade UI's}
    \noindent Before reading about these UI's it will come in \emph{handy} if you check out the \colorLink{noteColor}{Section:Player Data}{Player Data} section.
    \vspace{2mm}

    \noindent The store UI should allow the player to purchase the listed items:
    \vspace{3mm}

    \begin{itemize}
        \item \itemizeListDisplay{In game ``loot crates''}
        \item \itemizeListDisplay{Gamepass's}
        \item \itemizeListDisplay{Dev Products}
    \end{itemize}

    \vspace{2mm}
    \noindent To learn more about all of these please click \colorLink{noteColor}{Section:Monetization}{here}.\newline{}
    \noindent The Shop UI should offer the player options to select what they wanna buy, on the far right side it should
    display the item the player selected AND \emph{Show the items description}, it should also show you the items price,
    and the currency you would need to pay for the item.

    \vspace{2mm}
    \begin{spacing}{1.2}
        \noindent Once the player press's the \emph{trade button}, they will be promted with a player list that they can requests trades with.
        Upon sending a request if the player accept the request a new UI would be represented that will let the players put stuff from their inventory to the trading UI.
        The trading UI will be a giant frame, on the top of the frame will be a $6\times2$ grid, the first 6 slots will be for the first player and the other 6 for the other player.
        If they have multiple of the item they would be promoted with an option to select how much of \emph{that one item} should they put into the trade, this means they can have up to 6 \textcolor{importantColor}{\emph{unique}} items in the trade.
        On the right side of the main UI will be the other persons character frame with their name below them, and right below that would be a private chat between the two players.
        Below the trading part would be the players inventory with a filter bar on the top of that inventory frame, and right below the inventory will be an \textcolor{moneyColor}{accept} and \textcolor{red}{decline} button.    
    \end{spacing}
    \linkSection{Player Data}

    \linkSection{Monetization}
\end{document}